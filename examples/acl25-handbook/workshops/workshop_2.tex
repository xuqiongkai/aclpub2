\clearpage



\section[W02: The 12th Workshop on Argument Mining]{Workshop 02}
\label{workshop_2}

\begin{center}
    {\Large \textbf{W02: The 12th Workshop on Argument Mining}}\\
    \\


    Elena Chistova, Philipp Cimiano, Shohreh Haddadan, \\
    Gabriella Lapesa, Ramon Ruiz-Dolz \\                                                                                                                                                                                            

    Thursday, July 31
    
    Level 2 Hall B

\end{center}
	
Description: Argument Mining (also known as “argumentation mining”) is an emerging research area within computational linguistics that started with focusing on automatically identifying and classifying argument elements, covering several text genres such as legal documents, news articles, online debates, scholarly data, and many more. In recent years, the field (broadly Computational Argumentation) has grown to explore argument quality and synthesis on many levels. The field offers practical uses such as argument-focused search and debating technologies, e.g., IBM Project Debater. The growing interest in computational argumentation has led to several tutorials at major NLP conferences.

Besides providing a forum to discuss and exchange cutting edge research in this field, a secondary goal of this year's edition will be to broaden the disciplinary scope of the workshop by inviting other disciplines (e.g., (computational) social and political science, psychology, humanities) as well as other subareas of NLP to actively participate in the workshop and further shaping the field of argument mining. In particular, we would like to create synergies between the fields of argument mining and natural language reasoning.	
