\chapter{Workshops}

\begingroup
\renewcommand\arraystretch{1.8}

\section*{Thursday, July 31, 2025}

\begin{tabularx}{\textwidth}{>{\raggedright}X r}
    % \multicolumn{2}{l}{\textbf{Thursday, July 31, 2025}} \\

    \textbf{W01: 6th AfricaNLP Workshop: Multilingual and Multicultural-aware LLMs &
    \textit{Level 1 1.34} \\

    \textbf{W02: The 12th Workshop on Argument Mining} &
    \textit{Level 2 Hall B} \\

    \textbf{W03: 2nd Workshop on Natural Language Processing meets Climate Change} &
    \textit{Level 2 2.17} \\

    \textbf{W04: Eighth Workshop on Fact Extraction and VERification (FEVER)} &
    \textit{Level 2 2.31} \\

    \textbf{W05:  GEM2 Workshop: Generation, Evaluation & Metrics} &
    \textit{Level 2 Hall C} \\

    \textbf{W06: The 19th Linguistic Annotation Workshop (LAW XIX)} &
    \textit{Level 1 1.15-16} \\

    \textbf{W07: The 4th Workshop on NLP for Positive Impact} &
    \textit{Level 1 Hall N1} \\

    \textbf{W08: REALM: First Workshop for Research on Agent Language Models} &
    \textit{Level 1 1.61-62} \\

    \textbf{W09: SDP 2025: The 5th Workshop on Scholarly Document Processing} &
    \textit{Level 1 1.14} \\

    \textbf{W10: The Third Workshop on Social Influence in Conversations (SICon 2025)} &
    \textit{Level 1 1.31-32} \\

    \textbf{W11: Slavic NLP-2025: The 10th Biennial Workshop on Slavic NLP} &
    \textit{Level 2 2.44} \\

    \textbf{W12: The 4th Table Representation Learning Workshop at ACL 2025} &
    \textit{Level 2 2.15} \\
\end{tabularx}

\clearpage

\section*{Thursday, July 31 & Friday, August 1, 2025}

\begin{tabularx}{\textwidth}{>{\raggedright}X r}
    % \multicolumn{2}{l}{\textbf{Thursday, July 31 & Friday, August 1, 2025}} \\

    \textbf{W13: The Fourth Ukrainian Natural Language Processing Workshop (UNLP 2025)} &
    \textit{Gather} \\

    \textbf{W14: The 20th Workshop on Innovative Use of NLP for Building Educational Applications (BEA 2025)} &
    \textit{Level 1 1.85-86} \\

    \textbf{W15: 29th Conference on Computational Natural Language Learning (CoNLL)} &
    \textit{Level 1 Hall M1} \\

    \textbf{W16: The 22nd International Conference on Spoken Language Translation (IWSLT 2025)} &
    \textit{Level 1 Hall N2} \\

    \textbf{W17: The 19th Workshop on Semantic Evaluation (Semeval-2025)} &
    \textit{Level 1 Hall M2} \\

\end{tabularx}


\section*{Friday, August 1, 2025}

\begin{tabularx}{\textwidth}{>{\raggedright}X r}
    % \multicolumn{2}{l}{\textbf{Friday, August 1, 2025}} \\

    \textbf{W18: The Second Workshop on Analogical Abstraction in Cognition, Perception, and Language (Analogy-ANGLE II)} &
    \textit{Level 1 1.34} \\

    \textbf{W19: BioNLP 2025 and Shared Tasks (BioNLP-ST 2025)} &
    \textit{Level 2 2.15} \\

    \textbf{W20: The 6th Workshop on Gender Bias in Natural Language Processing (GeBNLP 2025)} &
    \textit{Level 2 Hall C} \\

    \textbf{W21: The First Workshop on Large Language Model Memorization (L2M2)} &
    \textit{Level 1 1.31-32} \\

    \textbf{W22: First Workshop on LLM Security (LLMSEC)} &
    \textit{Level 2 Hall B} \\

    \textbf{W23: "MAGMaR: Multimodal Augmented Generation via MultimodAl Retrieval 
Third Workshop on Multimodal Content Analysis for Social Good: MM4SG 2025"} &
    \textit{Level 2 2.44} \\

    \textbf{W24: FieldMatters–SIGTYP: The 7th Workshop on Research in Computational Linguistic Typology and Multilingual NLP (SIGTYP) 
and the Fourth Workshop on NLP Applications to Field Linguistics (FieldMatters)} &
    \textit{Level 1 Hall N1} \\

    \textbf{W25: The 3rd workshop on Towards Knowledgeable Foundation Models} &
    \textit{Level 1 1.14} \\

    \textbf{W26: WikiNLP: Advancing Natural Language Processing for Wikipedia } &
    \textit{Level 2 2.31} \\

    \textbf{W27: The 9th Workshop on Online Abuse and Harms (WOAH 2025) } &
    \textit{2.17} \\

    \textbf{W28: The First Joint Workshop on Large Language Models and Structure Modeling (XLLM)
LS+LM: Leveraging Linguistic Structures to Improve and Understand Language Models"} &
    \textit{Level 1 1.61-62} \\

\end{tabularx}

\clearpage



\section[W01: 6th AfricaNLP Workshop: Multilingual and Multicultural-aware LLMs]{Workshop 01}

\begin{center}
    {\Large \textbf{W01: 6th AfricaNLP Workshop: Multilingual and Multicultural-aware LLMs}}

    David Adelani, Constantine Lignos, Tajuddeen Gwadabe, Bonaventure F. P. Dosso, Israel Abebe Azime, Everlyn Asiko Chimoto, Clemencia Siro, Henok Biadglign Ademtew , Happy Buzaaba , Hatem Haddad, Menno Van Zaanen , Mmasibidi Setaka , Andiswa Bukula , Nomsa Skosana

    Thursday, July 31
    
    Level 1 Hall 1.34

\end{center}

Description: The AfricaNLP workshop brings together a diverse group of researchers to explore solutions, collaborations, and innovation around enhancing LLMs’ capabilities in African languages and ensuring cultural awareness in their applications. 

\clearpage



\section[W02: The 12th Workshop on Argument Mining]{Workshop 02}

\begin{center}
    {\Large \textbf{W02: The 12th Workshop on Argument Mining}}

    Elena Chistova, Philipp Cimiano, Shohreh Haddadan, Gabriella Lapesa, Ramon Ruiz-Dolz                                                                                                                                                                                                

    Thursday, July 31
    
    Level 2 Hall B

\end{center}
	
Description: Argument Mining (also known as “argumentation mining”) is an emerging research area within computational linguistics that started with focusing on automatically identifying and classifying argument elements, covering several text genres such as legal documents, news articles, online debates, scholarly data, and many more. In recent years, the field (broadly Computational Argumentation) has grown to explore argument quality and synthesis on many levels. The field offers practical uses such as argument-focused search and debating technologies, e.g., IBM Project Debater. The growing interest in computational argumentation has led to several tutorials at major NLP conferences.

Besides providing a forum to discuss and exchange cutting edge research in this field, a secondary goal of this year's edition will be to broaden the disciplinary scope of the workshop by inviting other disciplines (e.g., (computational) social and political science, psychology, humanities) as well as other subareas of NLP to actively participate in the workshop and further shaping the field of argument mining. In particular, we would like to create synergies between the fields of argument mining and natural language reasoning.	


\clearpage



\section[W03: 2nd Workshop on Natural Language Processing meets Climate Change]{Workshop 03}

\begin{center}
    {\Large \textbf{W03: 2nd Workshop on Natural Language Processing meets Climate Change}}

    Dominik Stammbach, Tobias Schimanski, Jingwei Ni, Alba Sun, Alok Singh, Christopher Manning, Gaku Morio, Kalyan Dutia, Peter Henderson, Saeid Vaghefi, Veruska Muccione

    Thursday, July 31
    
    Level 2 Hall 2.17

\end{center}

Description: TO add


\clearpage


\section[W04: Eighth Workshop on Fact Extraction and VERification (FEVER)]{Workshop 04}

\begin{center}
    {\Large \textbf{W04: Eighth Workshop on Fact Extraction and VERification (FEVER)}}

    Andreas Vlachos, Mubashara Akhtar, Michael Schlichtkrull, Christos Christodoulopoulos, Yulong Chen, Zhenyun Deng, Marek Strong, Arpit Mittal, Oana Cocarascu, Chenxi Whitehouse, James Thorne, Zhijiang Guo, Rami Aly, Rui Cao

     Thursday, July 31
    
    Level 2 Hall 2.31

\end{center}

Description: The Fact Extraction and Verification Workshop brings together researchers working on this topic as well as related ones such as recognizing textual entailment, question answering and argumentation mining.

\clearpage


\section[W05: Workshop on Meaningful, Efficient, and Robust Evaluation of LLMs and GEM: Natural Language Generation, Evaluation, and Metrics]{Workshop 05}

\begin{center}
    {\Large \textbf{W05: Workshop on Meaningful, Efficient, and Robust Evaluation of LLMs and GEM: Natural Language Generation, Evaluation, and Metrics}}

    Sebastian Gehrmann, Gabriel Stanovsky, Enrico Santus, Itay Itzhak, João Sedoc, Kaustubh Dhole, Michal Shmueli Scheuer, Miruna Clinciu, Ofir Arviv, Rotem Dror, Simon Mille, Yotam Perlitz, Oyvind Tafjord

    Thursday, July 31
    
    Level 2 Hall C

\end{center}

Description: Evaluating large language models (LLMs) is challenging. Running LLMs over medium or large scale corpus can be prohibitively expensive; they are consistently shown to be highly sensitive to prompt phrasing, and it is hard to formulate metrics which differentiate and rank different LLMs in a meaningful way. Consequently, the validity of the results obtained over popular benchmarks such as HELM or MMLU, lead to brittle conclusions (Sclar er al., 2024, Mizrahi et al., 2024, Alzahrani et al., 2024). We believe that meaningful, efficient, and robust evaluation is one of the cornerstones of the scientific method, and that achieving it should be a community-wide goal.

In this workshop we seek innovative research relating to the evaluation of LLMs and language generation systems in general. This includes, but is not limited to, robust, reproducible and efficient evaluation metrics, as well as new approaches for collecting evaluation data which can help in better differentiating between different systems and understanding their current bottlenecks.


\clearpage


\section[W06: The 19th Linguistic Annotation Workshop (LAW XIX)]{Workshop 06}

\begin{center}
    {\Large \textbf{W06: The 19th Linguistic Annotation Workshop (LAW XIX)}}

    Siyao Peng, Ines Rehbein, Amir Zeldes

    Thursday, July 31
    
   Level 1 Hall 1.15-16

\end{center}

Description: The Linguistic Annotation Workshop (LAW 2025) targets innovative research on linguistic annotation, focusing on subjectivity and the annotation of concepts from social sciences.

\clearpage


\section[W07: The 4th Workshop on NLP for Positive Impact]{Workshop 07}

\begin{center}
    {\Large \textbf{W07: The 4th Workshop on NLP for Positive Impact}}

    Ruyuan Wan, Katherine Atwell, Laura Biester, Angana Borah, Daryna Dementieva, Oana Ignat, Neema Kotonya, Ziyi Liu, Steven Wilson, Jieyu Zhao

   Thursday, July 31
    
    Level 1 Hall N1

\end{center}

Description: The 4th Workshop on NLP for Positive Impact aims to advance innovative NLP research for positive societal impact, emphasizing responsible methods and diverse applications.


\clearpage



\section[W08: REALM: First Workshop for Research on Agent Language Models]{Workshop 08}

\begin{center}
    {\Large \textbf{W08: REALM: First Workshop for Research on Agent Language Models}}

    Ehsan Kamalloo, Nicolas Gontier, Xing Han Lu, Shikhar Murty, Alexandre Lacoste, Nouha Dziri, Hanna Hajishirzi, Graham Neubig

    Thursday, July 31
    
    Level 1 Hall 1.61-62

\end{center}

Description: The REALM workshop aims to bring together researchers, practitioners, and thought leaders to discuss and align on the current landscape, key challenges, and future directions of LLM Agents.


\clearpage


\section[W09: SDP 2025: The 5th Workshop on Scholarly Document Processing]{Workshop 09}

\begin{center}
    {\Large \textbf{W09: SDP 2025: The 5th Workshop on Scholarly Document Processing}}

    Tirthankar Ghosal, Philipp Mayr, Aakanksha Naik, Amanpreet Singh, Anita de Waard, Dayne Freitag, Georg Rehm, Sonja Schimmler, Dan Li

    Thursday, July 31
    
    Level 1 Hall 1.14

\end{center}

Description: The goal of this workshop is to provide a venue for addressing these challenges, as well as a platform for tasks and resources supporting the processing of scientific documents. Our long-term objective is to establish scholarly and scientific texts as an essential domain for NLP research, to supplement current efforts on web text and news articles.

\clearpage


\section[W10: The Third Workshop on Social Influence in Conversations (SICon 2025)]{Workshop 10}

\begin{center}
    {\Large \textbf{W10: The Third Workshop on Social Influence in Conversations (SICon 2025)}}

    James Hale, Deuksin Kwon, Kushal Chawla, Ritam Dutt, Muskan Garg, Liang Qiu, Alexandros Papangelis, Gale Lucas, Zhou Yu, Daniel Hershcovich

   Thursday, July 31
    
    Level 1 Hall 1.31-32

\end{center}
	
Description: Social influence (SI) is the change in an individual's thoughts, feelings, attitudes, or behaviors from interacting with another individual or a group. For example, a buyer uses SI skills to negotiate trade-offs and build rapport with the seller. SI is ubiquitous in everyday life, and hence, realistic human-machine conversations must reflect these dynamics, making it essential to model and understand SI in dialogue research systematically. This would improve SI systems' ability to understand users’ utterances, tailor communication strategies, personalize responses, and actively lead conversations. These challenges draw on perspectives not only from NLP and AI research but also from Game Theory, Affective Computing, Communication, and Social Psychology.


\clearpage



\section[W11: Slavic NLP-2025: The 10th Biennial Workshop on Slavic NLP]{Workshop 11}

\begin{center}
    {\Large \textbf{W11: Slavic NLP-2025: The 10th Biennial Workshop on Slavic NLP}}

   Jakub Piskorski, Michał Marcińczuk, Preslav Nakov, Nikola Ljubešić, Pavel Přibáň, Roman Yangarber

    Thursday, July 31
    
   Level 2 Hall 2.44

\end{center}

Description:

\clearpage



\section[W12: The 4th Table Representation Learning Workshop at ACL 2025]{Workshop 12}

\begin{center}
    {\Large \textbf{W12: The 4th Table Representation Learning Workshop at ACL 2025}}

   Madelon Hulsebos, Shuaichen Chang, Wenhu Chen, Filip Gralinski, Qian Liu, Huan Sun

    Thursday, July 31
    
   Level 2 Hall 2.15

\end{center}

Description: "Tables are a promising modality for representation learning and generative models with too much application potential to ignore. The Table Representation Learning (TRL) workshop is the premier venue in this emerging research area and has three main goals:
(1) Motivate structured data (e.g. tables) as a primary modality for representation and generative models and advance the area further.
(2) Showcase impactful applications of pretrained table models and identify open challenges for future research, with a particular focus on progress in NLP for this edition at ACL in 2025.
(3) Foster discussion and collaboration across the NLP, ML, IR and DB communities."


\clearpage



\section[W13: The Fourth Ukrainian Natural Language Processing Workshop (UNLP 2025)]{Workshop 13}

\begin{center}
    {\Large \textbf{W13: The Fourth Ukrainian Natural Language Processing Workshop (UNLP 2025)}}

    Mariana Romanyshyn, Olena Nahorna, Oleksii Ignatenko, Andrii Hlybovets
    
    Thursday, July 31 and Friday, August 1, 2025
    
    Virtually on Gather

\end{center}

Description: The UNLP workshop is dedicated to the development of language resources, tools, and NLP solutions for the Ukrainian language. This event brings together professionals from academia and industry who work with Ukrainian or do cross-lingual research that can be applied to Ukrainian. Every year UNLP hosts a shared task and invited speakers who promote research on low-to-mid-resource languages.

\clearpage



\section[W14: The 20th Workshop on Innovative Use of NLP for Building Educational Applications (BEA 2025)]{Workshop 14}

\begin{center}
    {\Large \textbf{W14: The 20th Workshop on Innovative Use of NLP for Building Educational Applications (BEA 2025)}}

    Ekaterina Kochmar, Marie Bexte, Andrea Horbach, Ronja Laarmann-Quante, Anaïs Tack, Bashar Alhafni, Victoria Yaneva, Jill Burstein, Zheng Yuan
    
    Thursday, July 31 and Friday, August 1, 2025
    
    Level 1 Hall 1.85-86

\end{center}

Description: The BEA Workshop is a leading venue for NLP innovation in the context of educational applications.

\clearpage


\section[W15: 29th Conference on Computational Natural Language Learning (CoNLL)]{Workshop 15}

\begin{center}
    {\Large \textbf{W15: 29th Conference on Computational Natural Language Learning (CoNLL)}}

    Gemma Boleda, Michael Roth, Emily Cheng, Selina Meyer, Snigdha Chaturvedi
    
    Thursday, July 31 and Friday, August 1, 2025
    
    Level 1 Hall M1

\end{center}

Description: CoNLL is a yearly conference organized by SIGNLL (ACL's Special Interest Group on Natural Language Learning), focusing on theoretically, cognitively, and scientifically motivated approaches to computational linguistics.  In particular, it seeks to explore the interaction between theoretical issues in linguistics and cognition, on the one hand, and computational modeling, on the other.

\clearpage



\section[W16: The 22nd International Conference on Spoken Language Translation (IWSLT 2025)]{Workshop 16}

\begin{center}
    {\Large \textbf{W16: The 22nd International Conference on Spoken Language Translation (IWSLT 2025)}}

   Marcello Federico, Alex Waibel, Elizabeth Salesky, Jan Niehues, Sebastian Stüker, Atul Kr. Ojha, Marine Carpuat
    
    Thursday, July 31 and Friday, August 1, 2025
    
    Level 1 Hall N2

\end{center}

Description: "The International Conference on Spoken Language Translation (IWSLT) is an annual scientific conference dedicated to all aspects of spoken language translation. For more than 20 years, the conference has published research papers and organized key evaluation campaigns within the field, including the creation of key data suites, benchmarks, metrics and new tasks. IWSLT is the annual meeting of SIGSLT, the ACL-ISCA-ELRA Special Interest Group on Spoken Language Translation."

\clearpage



\section[W17: The 19th Workshop on Semantic Evaluation (Semeval-2025)]{Workshop 17}

\begin{center}
    {\Large \textbf{W17: The 19th Workshop on Semantic Evaluation (Semeval-2025)}}

   Sara Rosenthal, Aiala Rosá, Marcos Zampieri, Debanjan Ghosh
    
    Thursday, July 31 and Friday, August 1, 2025
    
    Level 1 Hall M2

\end{center}

Description: "The Semantic Evaluation (SemEval) workshops focus on the evaluation and comparison of systems that
analyze diverse semantic phenomena in text, with the aim of extending the current state of the art in
semantic analysis and creating high quality annotated datasets in a range of increasingly challenging
problems in natural language semantics."

\clearpage



\section[W18: The Second Workshop on Analogical Abstraction in Cognition, Perception, and Language (Analogy-ANGLE II)]{Workshop 18}

\begin{center}
    {\Large \textbf{W18: The Second Workshop on Analogical Abstraction in Cognition, Perception, and Language (Analogy-ANGLE II)}}

  Filip Ilievski, Giulia Rambelli, Marianna Bolognesi, Ute Schmid, Pia Sommerauer

    Friday, August 1, 2025
    
   Level 1 Hall 1.34
\end{center}

Description: Analogy-Angle II
A second edition of an interdisciplinary workshop co-located with ACL 2025. Analogy-Angle II will occur in Vienna, Austria on August 1, 2025.

The Second Workshop on Analogical Abstraction in Cognition, Perception, and Language (Analogy-Angle II)
Explore, model, and understand analogical reasoning in cognition, language, and computational models from an interdisciplinary perspective

Analogy-Angle II is a multidisciplinary workshop to advance research on analogical abstraction by bridging the fields of computational linguistics, artificial intelligence, and cognitive psychology. This workshop seeks to foster collaboration among researchers by providing a platform for sharing novel insights, benchmarks, methodologies, and analogy applications across disciplines.


\clearpage


\section[W19: BioNLP 2025 and Shared Tasks (BioNLP-ST 2025)]{Workshop 19}

\begin{center}
    {\Large \textbf{W19: BioNLP 2025 and Shared Tasks (BioNLP-ST 2025)}}

 Dina Demner-Fushman, Sophia Ananiadou, Makoto Miwa, Jun-ichi Tsujii

    Friday, August 1, 2025
    
  Level 2 Hall 2.15
\end{center}

Description: BioNLP associated with ACL SIGBIOMED is an established primary venue for presenting research in language processing for the biomedical domains. The workshop has been running every year since 2002 and continues getting stronger. BioNLP truly encompasses the breadth of the domain and brings together researchers in biomedical and clinical NLP from all over the world.

\clearpage


\section[W20: The 6th Workshop on Gender Bias in Natural Language Processing (GeBNLP 2025)]{Workshop 20}

\begin{center}
    {\Large \textbf{W20: The 6th Workshop on Gender Bias in Natural Language Processing (GeBNLP 2025)}}

Christine Basta, Marta R. Costa-jussà, Agnieszka Falénska, Debora Nozza, Karolina Stańczak
    Friday, August 1, 2025
    
 Level 2 Hall C
\end{center}

Description: The GeBNLP workshop provides a dedicated forum to address gender and other demographic biases in natural language processing models. It aims to foster awareness, share research on mitigating bias through data and algorithmic approaches, and build community consensus around standard evaluation tasks and metrics.

\clearpage


\section[W21: The First Workshop on Large Language Model Memorization (L2M2)]{Workshop 21}

\begin{center}
    {\Large \textbf{W21: The First Workshop on Large Language Model Memorization (L2M2)}}

Robin Jia, Eric Wallace, Yangsibo Huang, Tiago Pimentel, Pratyush Maini, Verna Dankers, Johnny Wei, Pietro Lesci
    Friday, August 1, 2025

    Level 1 Hall 1.31-32
    
\end{center}

Description: The First Workshop on Large Language Model Memorization (L2M2), co-located with ACL 2025 in Vienna, seeks to provide a central venue for researchers studying LLM memorization from these different angles.

\clearpage


\section[W22: First Workshop on LLM Security (LLMSEC)]{Workshop 22}

\begin{center}
    {\Large \textbf{W22: First Workshop on LLM Security (LLMSEC)}}

Leon Derczynski, Jekaterina Novikova, Muhao Chen
    Friday, August 1, 2025

   Level 2 Hall B
    
\end{center}

Description: Work on adversarially-induced failure modes of large language models, the conditions that lead to them, and their mitigations.

\clearpage


\section[W23: MAGMaR: Multimodal Augmented Generation via MultimodAl Retrieval ]{Workshop 23}

\begin{center}
    {\Large \textbf{W23: MAGMaR: Multimodal Augmented Generation via MultimodAl Retrieval }}

Reno Kriz, Kenton Murray, Kate Sanders, Eugene Yang, Cameron Carpenter, Francis Ferraro, Benjamin Van Durme
    Friday, August 1, 2025

  Level 2 Hall 2.44
    
\end{center}

Description: Vast amounts of information today is being stored as videos with minimal text metadata, necessitating further research around the efficient discovery, understanding, and synthesis of multimodal collections. To address this need, the workshop on Multimodal Augmented Generation via Multimodal Retrieval (MAGMaR) workshop will focus on two primary areas: (1) the retrieval of multimodal content, which spans text, images, audio, video, and multimodal data (e.g., image-language, video-language); and (2) retrieval-augmented generation, with an emphasis on multimodal retrieval and generation.

\clearpage


\section[W24: FieldMatters–SIGTYP: The 7th Workshop on Research in Computational Linguistic Typology and Multilingual NLP (SIGTYP) and the Fourth Workshop on NLP Applications to Field Linguistics (FieldMatters)]{Workshop 24}

\begin{center}
    {\Large \textbf{W24: FieldMatters–SIGTYP: The 7th Workshop on Research in Computational Linguistic Typology and Multilingual NLP (SIGTYP) and the Fourth Workshop on NLP Applications to Field Linguistics (FieldMatters)}}

Michael Hahn, Ekaterina Vylomova

    Friday, August 1, 2025

 Level 1 Hall N1
    
\end{center}

Description: SIGTYP is the first dedicated venue for typology-related research and its integration in multilingual NLP.

\clearpage


\section[W25: The 3rd workshop on Towards Knowledgeable Foundation Models]{Workshop 25}

\begin{center}
    {\Large \textbf{W25: The 3rd workshop on Towards Knowledgeable Foundation Models}}

Yuji Zhang, Xiaozhi Wang, Mor Geva, Chi Han, Shangbin Feng, Silin Gao, Sha Li, Manling Li, Heng Ji

    Friday, August 1, 2025

 Level 1 Hall N1
    
\end{center}

Description: Knowledge has been an important pre-requisite for a variety of AI applications, and is typically sourced from either structured knowledge sources such as knowledge bases and dictionaries or unstructured knowledge sources such as Wikipedia documents. More recently, researchers have discovered that language models already possess a significant amount of knowledge through pre-training: LLMs can be used to generate commonsense knowledge and factual knowledge context for question answering. While the results are encouraging, there are still lingering questions: Where does this knowledge come from? How much do language models know? Is this knowledge reliable? If some knowledge is wrong, can we fix it?

\clearpage


\section[W26: WikiNLP: Advancing Natural Language Processing for Wikipedia]{Workshop 26}

\begin{center}
    {\Large \textbf{W26: WikiNLP: Advancing Natural Language Processing for Wikipedia}}

Akhil Arora, Isaac Johnson, Lucie-Aimée Kaffee, Tzu-Sheng Kuo, Tiziano Piccardi, Indira Sen

    Friday, August 1, 2025

Level 2 Hall 2.31
    
\end{center}

Description: WikiNLP invites researchers to contribute novel uses of Wikimedia data or studies of the impact of Wikimedia data within the NLP community. We will discuss tensions around multilinguality and concerns raised by generative AI. We will also highlight successful approaches to developing tooling that can assist the Wikimedia community in maintaining and improving the breadth of the Wikimedia projects.

\clearpage


\section[W27: The 9th Workshop on Online Abuse and Harms (WOAH 2025) ]{Workshop 27}

\begin{center}
    {\Large \textbf{W27: The 9th Workshop on Online Abuse and Harms (WOAH 2025) }}

Zeerak Talat, Agostina Calabrese, Christine de Kock, Francielle Vargas, Flor Miriam Plaza del Acro, Debora Nozza

    Friday, August 1, 2025

Level 2 Hall 2.17

\end{center}

Description: WOAH, the 9th Workshop on Online Abuse and Harms, invites paper submissions from a diverse range of fields. Digital technologies have brought significant benefits to society, transforming how people connect, communicate, and interact. However, these same technologies have also enabled the widespread dissemination and amplification of abusive and harmful content, such as hate speech, harassment, and misinformation. Given the sheer volume of content shared online, addressing abuse and harm at scale requires the use of computational tools. Yet, detecting and moderating online abuse remains a complex task, fraught with technical, social, legal, and ethical challenges. 

\clearpage


\section[W28: The First Joint Workshop on Large Language Models and Structure Modeling (XLLM)
LS+LM: Leveraging Linguistic Structures to Improve and Understand Language Models"]{Workshop 28}

\begin{center}
    {\Large \textbf{W28: The First Joint Workshop on Large Language Models and Structure Modeling (XLLM)
LS+LM: Leveraging Linguistic Structures to Improve and Understand Language Models"}}

Zeerak Talat, Agostina Calabrese, Christine de Kock, Francielle Vargas, Flor Miriam Plaza del Acro, Debora Nozza

    Friday, August 1, 2025

Level 1 Hall 1.61-62

\end{center}

Description: The 1st Joint Workshop on Large Language Models and Structure Modeling (XLLM 2025) at ACL 2025 aims to encourage discussions and highlight methods for language structure modeling in the era of LLMs.  
