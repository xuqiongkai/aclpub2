\begin{center}
    \Large{\textbf{T8: Guardrails and Security for LLMs: Safe, Secure, and Controllable Steering of LLM Applications}\\}
    \par\bigskip
    \large{Traian Rebedea, Leon Derczynski, Shaona Ghosh, Makesh Narsimhan Sreedhar,
Faeze Brahman, Liwei Jiang, Bo Li, Yulia Tsvetkov, Christopher Parisien and Yejin Choi}\\
    \par\bigskip

\end{center}

Pretrained generative models, especially large language models, provide novel ways for users to interact with computers. While generative NLP research and applications had previously aimed at very domain-specific or task-specific solutions, current LLMs and applications (e.g. dialogue systems, agents) are versatile across many tasks and domains. Despite being trained to be helpful and aligned with human preferences (e.g., harmlessness), enforcing robust guardrails on LLMs remains a challenge. And, even when protected against rudimentary attacks, just like other complex software, LLMs can be vulnerable to attacks using sophisticated adversarial inputs. This tutorial provides a comprehensive overview of key guardrail mechanisms developed for LLMs, along with evaluation methodologies and a detailed security assessment protocol - including auto red-teaming of LLM-powered applications. Our aim is to move beyond the discussion of single prompt attacks and evaluation frameworks towards addressing how guardrailing can be done in complex dialogue systems that employ LLMs.

We aim to provide a cutting-edge and complete overview of deployment risks associated with LLMs in production environments. While the main focus will be on how to effectively protect against safety and security threats, we also tackle the more recent topic of providing dialogue and topical rails, including respecting custom policies. We also examine the new attack vectors introduced by LLM-enabled dialogue systems, such as methods for circumventing dialogue steering.
\begin{center}
    \noindent\rule{200px}{1pt}
\end{center}
