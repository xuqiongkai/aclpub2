\begin{center}
    \Large{\textbf{"T4: Human-AI Collaboration: How AIs Augment Human Teammates"}\\}
    \par\bigskip
    \large{Tongshuang Wu, Diyi Yang, Kyle Lo and Marti A. Hearst}\\
    \par\bigskip

\end{center}

The continuous, rapid development of general- purpose models like LLMs suggests the theoretical possibility of AI performing any human task. 
Yet, despite the potential and promise, these models are far from perfect, excelling at certain tasks while struggling with others. 
The tension between what is possible and a model's limitations raises the general research question that has attracted attention from various disciplines: What is the best way to use AI to maximize its benefits? In this tutorial, we will review recent developments related to human-AI teaming and collaboration. 
To the best of our knowledge, our tutorial will be the first to provide a more integrated view from NLP, HCI, Computational Social Science, and Learning Science, etc., and highlight how different communities have identified the goals and societal impacts of such collaborations, both positive and negative. We will further discuss how to operationalize these Human-AI collaboration goals, and reflect on how state-of-the-art AI models should be evaluated and scaffolded to make them most useful in collaborative contexts.
\begin{center}
    \noindent\rule{200px}{1pt}
\end{center}
