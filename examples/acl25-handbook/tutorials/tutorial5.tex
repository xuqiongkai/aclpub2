\begin{center}
    \Large{\textbf{T5: Navigating Ethical Challenges in NLP: Hands-on strategies for students and researchers}\\}
    \par\bigskip
    \large{Luciana Benotti, Fanny Ducel, Karen Fort, Guido Ivetta, Zhijing Jin, Min-Yen Kan,
Seunghun Lee, Margot Mieskes, Minzhi Li, and Adriana Pagano}\\
    \par\bigskip

\end{center}

With NLP research being rapidly productionized into real-world applications, it is important to be aware of and think through the consequences of our research. Such ethical considerations are important in both authoring and reviewing (e.g. privacy, consent, fairness, etc).This tutorial will equip participants with ba-sic guidelines for thinking deeply about ethical issues and review common considerations that recur in NLP research. The methodology is interactive and participatory, including case studies and working in groups. Participants will gain practical experience on when to flag a paper for ethics review and how to write an ethical consideration section, which will be shared with the broader community. Importantly, the participants will be co-creating the tutorial outcomes and extending tutorial materials to share as public outcomes.

\begin{center}
    \noindent\rule{200px}{1pt}
\end{center}
