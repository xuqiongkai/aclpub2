\textbf{Message from the Program Chairs}\\
\\

Welcome to the 63rd Annual Meeting of the Association for Computational Linguistics!

ACL 2025 will be held in a hybrid format, offering attendees the option to join us in person in vibrant Vienna, Austria, or to participate remotely from anywhere in the world. Organizing ACL 2025 has been a collaborative effort, made possible by the dedication and hard work of thousands of people. We gratefully acknowledge the support and contributions of the following people:

\begin{itemize}
\item The General Chair, Roberto Navigli;
\item The ARR Editors-in-Chief of the February 2025 cycle (Jun Suzuki, Jing Jiang, and Xiaodan Zhu) and the entire team (Mausam, Viviane Moreira, Vincent Ng, Lilja Øvrelid, Anna Rogers, Michael White, Margot Mieskes, Sarvnaz Karimi);
\item The techical OpenReview chairs, Niket Tandon, Lizhen Qu, and the OpenReview support team, in particular Rachel, for multiple rounds of technical help in setting up ACL 2025 on the Open Review platform;
\item The 169 Senior Area Chairs;
\item The 1,937 Area Chairs and the 11,720  reviewers; 
\item The best paper committee chairs, Rada Mihalcea and Roi Reichart, and the best paper committee members; 
\item The ethics chairs, Karën Fort and Bjorn Ross; 
\item The workshop chairs, Terra Blevins and Christophe Gravier;
\item The tutorial chairs, Yuki Arase, David Jurgens and Fei Xia;
\item The industry track chairs, Yunyao Li and Georg Rehm;
\item The demonstration chairs, Pushkar Mishra, Smaranda Muresan, and Tao Yu;
\item The internal communications chairs, Sara Tonelli and Yiquan Wu;
\item The website and conference app chairs, Xudong Han and Alessandro Raganato; 
\item The publication chairs, Pierpaolo Basile, Libo Qin, and Zhenghao Liu;
\item The handbook chairs, Els Lefever and Qiongkai Xu;
\item The local organization chairs, Benjamin Roth and Dagmar Gromann, and their team;
\item The visa chairs, Rexhina Blloshmi and Eleni Ilkou;
\item The publicity and social media chairs, Anette Frank, Shruti Rijhwani and Horacio Saggion;
\item The documentation chair, Chenghua Lin;
\item The student research workshop chairs, Zhu Liu, Mingyang Wang and Jin Zhao;
\item The student research workshop chairs faculty advisors, Lea Frermann, Daniel Hershcovich               
and Tristan Miller;
\item The student volunteer chairs, Pedro Henrique Luz de Araujo and Eleonora Mancini;
\item The diversity and inclusion chairs, Senja Pollak, Maria Ryskina, Shane Storks and Hwaran Lee;
\item The sponsorship chairs, Raffaella Bernardi and Thomas Scialom;
\item The virtual infrastructure chairs, Manling Li, Yang Liu and Avi Sil;
\item The ACL Anthology Director Matt Post and his team;
\item The TACL editors-in-chief (Asli Celikyilmaz, Roi Reichart, Dilek Hakkani-Tur) and CL Editor in-Chief Wei Lu for coordinating TACL and CL presentations with us;
\item The ACL 2024 Program Chairs, Lun-Wei Ku, André F. T. Martins, Vivek Srikumar, for information and support;
\item Damira Mrsic and Underline Team; 
\item Jennifer Rachford and entire conference support staff; 
\item All the authors of papers who submitted their papers for review in the ARR 2025 February cycle and those who committed to the ACL 2025 conference.
\end{itemize}

\textbf{Review Process}
All submissions to ACL 2025 went through a two-stage review process. First, papers were submitted to the ACL Rolling Review (ARR), where they were reviewed by reviewers and received meta-reviews from Area Chairs. Then, authors had the option to commit their reviewed papers to ACL via a separate ACL 2025 commitment site. At this stage, Senior Area Chairs provided recommendations, and final acceptance decisions were made by the Program Chairs. This process is consistent with previous conferences, including ACL 2024, EACL 2024, and NAACL 2025.

We worked closely with the ARR team, particularly the ARR February 2025 Editors-in-Chief, and served as guest Editors-in-Chief for this round. We helped recruit new reviewers and Area Chairs to ARR, resulting in 11,720 reviewers and 1,942 Area Chairs in the February 2025 ARR cycle to which most ACL 2025 papers were submitted. ACL also recruited 169 Senior Area Chairs to oversee the review and meta-review process. Overall, the ARR process ran smoothly, ensuring that all submitted papers received at least three reviews and a meta-review. For the ACL commitment phase, Senior Area Chairs made recommendations for 5,356 committed papers based on the reviews, meta-reviews, and the papers themselves, with final acceptance decisions made by the Program Chairs.\\

\textbf{Acceptance Rate}

In total, there are 1,699 papers accepted to the Main Conference and 1,392 papers accepted to Findings. The acceptance rate calculation follows precedent set by previous conferences that go through ACL Rolling Review (ARR), e.g. ACL 2024. The calculation takes into account the multi-stage process of ARR where a paper may get revised in ARR and then later committed to the conference. In total, we had 8,360 unique submissions across the December 2024 and February 2025 ARR cycles of which 5,501 papers were committed to ACL. The acceptance rate is 20.3\% for the Main Conference papers and a further 16.7\% for Findings papers.\\

\textbf{Special Theme: Generalization of NLP Models}

ACL 2025's special theme is generalization of NLP models. Generalization is crucial for ensuring that models behave robustly, reliably, and fairly when making predictions on data different from their training data. Achieving good generalization is critically important for models used in real-world applications, as they should emulate human-like behavior. Humans are known for their ability to generalize well, and models should aspire to this standard. The theme track invites empirical and theoretical research and position and survey papers reflecting on the Generalization of NLP Models. The possible topics of discussion include (but are not limited to) the following:
1. How can we enhance the generalization of NLP models across various dimensions—compositional, structural, cross-task, cross-lingual, cross-domain, and robustness?
2. What factors affect the generalization of NLP models?
3. What are the most effective methods for evaluating the generalization capabilities of NLP models?
4. While Large Language Models (LLMs) significantly enhance the generalization of NLP models, what are the key limitations of LLMs in this regard?

We received 128 submissions to the theme track during the review phase. Among these, 41 papers were accepted to the main conference and a further 33 to Findings of ACL 2025. The conference will also feature a panel discussion on the theme of Generalization, with the participation of leading experts in this area.\\

\textbf{Best Paper Selection}

ACL 2025 implemented the updated ACL award policy that seeks to expand the pool
of work recognized as outstanding. In total 117 papers were nominated by the reviewers, area chairs and senior area chairs for best paper consideration. The best paper committee assessed these papers to select the best papers (featuring $(\leq)0.6\%$ of accepted papers), outstanding papers (featuring $(\leq)2.5\%$ of accepted papers), and special awards for social impact and best resource.  Based on the review by the Best Paper committee, 39 papers have been selected for awards in the above categories. Separately, the senior area chairs also nominated their favourite papers as SAC Highlights.

In addition, we have awards for test of time award for a paper published in TACL in 2013 or 2014 and the best paper award for a paper published in TACL in 2024. The final selection was made by the best paper committee, and the winners will be announced during the closing ceremony. The ACL 2025 Best Papers will also be given an opportunity to present their work in the closing ceremony.\\

\textbf{Program Composition \& Presentation Modes}

Based on feedback from the conference support staff and the Underline team after ACL 2025, we decided to hold the virtual presentations sessions during the main conference. This enables us to align the virtual sessions with time slots when in-person participants are available. This approach allows virtual attendees to participate concurrently with the physical event, avoiding the need for organizers and attendees to engage with the conference twice and separately.
This year, 218 main conference papers were selected for oral presentations by the program chairs, with the goal of creating a well-rounded program featuring a diverse set of topics instead of selecting papers based on their review scores. This year we have introduced a panel section of the conference. Out of the selected oral presenters, 25 will have an opportunity to not only present their work but also participate in a panel discussion. We have lined up five panels on the theme of: Generalisation in NLP, LLM alignment, Human-centred NLP, Interpretability and model analysis, Multilinguality and language diversity.
In addition to the main conference papers, the ACL program also includes 18 papers accepted by Computational Linguistics and 42 papers accepted by Transactions of the ACL (TACL). Among these, 11 journal papers will be presented in-person as oral presentations, thematically distributed across appropriate sessions. 

This year the conference will also for the first time feature a dedicated Findings poster session with reception. All Findings presenters have been assigned poster presentations in this session or alongside other posters for main conference papers in the same track.
Rounding out the program are dedicated sessions for the demonstrations track and the student research workshop.\\


\textbf{Keynotes and Panel}

This year’s program features an impressive lineup of two keynote presentations:
\begin{itemize}
\item Prof. Luke Zettlemoyer from Paul G. Allen School of Computer Science & Engineering at the University of Washington, and a Senior Research Director at Meta, will share his insights on “Rethinking Pretraining: Data and Architecture.”
\item Prof. Verena Rieser from Google DeepMind will present on “Whose Gold? Re-imagining Alignment for Truly Beneficial AI.”
\end{itemize}

Alongside these keynotes, we are thrilled to host a panel discussion aiming to answer the question Can large language models (LLMs) generalize?. Our esteemed panelists include:
\begin{itemize}
\item Prof. Eduard Hovy, University of Melbourne (who will also act as the panel chair).
\item Prof. Mirella Lapata, University of Edinburgh
\item Prof. Yue Zhang, Westlake University
\item Prof. Dan Roth, University of Pennsylvania and Oracle
\end{itemize}

This diverse group of panelists will provide a comprehensive view of the latest trends and challenges in Generalisation of NLP in the Large Language Models era. \\

We hope you enjoy this year’s diverse and engaging program!\\

\textbf{Ekaterina Shutova} (University of Amsterdam and Stanford University)\\
\textbf{Mohammad Taher Pilehvar} (Cardiff University and Tehran Institute for Advanced Studies)\\
\textbf{Joyce Nakatumba-Nabende} (Makerere University)\\
\textbf{Wanxiang Che} (Harbin Institute of Technology)\\

ACL 2025 Program Co-Chairs
