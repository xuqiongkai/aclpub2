\chapter{Anti-harassment Policy}
\vspace*{0.2cm}

ACL 2025 adheres to the ACL Anti-Harassment Policy. Any participant who experiences harassment or hostile behavior may contact any current member of the ACL Professional Conduct Committee or Jennifer Rachford, who is usually available at the registration desk of the conference. Please be assured that if you approach us, your concerns will be kept in strict confidence, and we will consult with you on any actions taken. The open exchange of ideas, the freedom of thought and expression, and respectful scientific debate are central to the aims and goals of a ACL conference. These require a community and an environment that recognizes the inherent worth of every person and group, that fosters dignity, understanding, and mutual respect, and that embraces diversity. For these reasons, ACL is dedicated to providing a harassment-free experience for participants at our events and in our programs. 
Harassment and hostile behavior are unwelcome at any ACL conference. This includes speech or behavior (including in public presentations and on-line discourse) that intimidates, creates discomfort, or interferes with a persons participation or opportunity for participation in the conference. We aim for ACL conferences to be an environment where harassment in any form does not happen, including but not limited to: harassment based on race, gender, religion, age, color, national origin, ancestry, disability, sexual orientation, or gender identity. 
Harassment includes degrading verbal comments, deliberate intimidation, stalking, harassing photography or recording, inappropriate physical contact, and unwelcome sexual attention. 
The ACL board members are listed at: 
https://www.aclweb.org/portal/about. 
The full policy and its implementation is defined at: 
https://aclweb.org/adminwiki/index.php?title=Anti\-Harassment\_Policy

\clearpage

\chapter{Ethics Policy}
\vspace*{0.2cm}

ACL adopts the ACM Code of Ethics (https://www.acm.org/code\-of\-ethics) in the version adopted June 22nd, 2018, by the ACM Council. 
In its application to ACL, it is to be read in the contextually appropriate interpretation, e.g., ACM member is to be read as ACL member. Sec 4.2 should be read as follows: 4.2 Treat violations of the Code as inconsistent with membership in the ACL. Each ACL member should encourage and support adherence by all members of the CL/NLP community regardless of ACL membership. ACL members who recognize a breach of the Code should consider reporting the violation to the ACL, which may result in remedial action.
The open exchange of ideas, the freedom of thought and expression, and respectful scientific debate are central to the aims and goals of the ACL. These require a community and an environment that recognizes the inherent worth of every person and group, that fosters dignity, understanding, and mutual respect, and embraces diversity. For these reasons, ACL is dedicated to providing a harassment-free experience for all the members, as well as participants at our events and in our programs. Harassment and hostile behavior are unwelcome at any ACL conference, associated event, or in ACLaffiliated online discussions. This includes speech or behavior that intimidates, creates discomfort, or interferes with a persons participation or opportunity for participation in a conference or an event. We aim for ACL-related activities to be an environment where harassment in any form does not happen, including but not limited to: harassment based on race, gender, religion, age, color, appearance, national origin, ancestry, disability, sexual orientation, or gender identity. Harassment includes degrading verbal comments, deliberate intimidation, stalking, harassing photography or recording, inappropriate physical contact, and unwelcome sexual attention. The policy is not intended to inhibit challenging scientific debate, but rather to promote it by ensuring that all are welcome to participate in the shared spirit of scientific inquiry. Vexatious complaints and willful misuse of this procedure will render the complainant subject to the same sanctions as a violation of the anti-harassment policy. It is the responsibility of the community as a whole to promote an inclusive and positive environment for our scholarly activities. In addition, anyone who experiences harassment or hostile behavior may contact any member of the Professional Conduct Committee. Members of this committee are instructed to keep any such contact in strict confidence, and those who approach the committee will be consulted before any actions are taken.
