\chapter{Local Guide}
\vspace*{0.2cm}

\section*{About Vienna}

Architecturally, the city is still characterized primarily by the buildings around Vienna’s Ringstrasse from the Wilhelminian period, but also by Baroque and Art Nouveau (Vienna Modernism and Vienna Secession). Through its role as the imperial capital and residence of the Austrian Empire from 1804 onwards, Vienna became a cultural and political center of Europe. The historic center of Vienna and Schönbrunn Palace are now UNESCO World Heritage sites. With around 7.5 million tourists annually and around 18.9 million overnight stays in 2024, Vienna is one of the most visited cities in Europe.


\section*{Language}

German is considered the official language of Austria, however English is widely spoken throughout the country.


\section*{Safety}

Vienna is generally considered a very safe place to visit, but like visiting any major city, be aware of your surroundings and of scams, know emergency numbers (listed below), keep digital copies of important documents (passport).

\section*{Power Plugs/Outlets}

Outlets in Vienna are European Type F and the voltage is 230 volts, 50 Hz. This means you'll need a travel adapter or possible converter for devices from countries like the US, which have different plug types and voltages. 

\section*{In case of an Emergency}

European Emergency Number: 112 (Ambulance, Fire, Police)

\section*{Clothing}

Plan on packing light, breathable summer clothing. Please note that in Vienna there can be very hot summer days and heavy rain.Water mists are provided at certain locations for cooling refreshment. The City of Vienna also provides free water from drinking fountains.

\section*{Currency & Living Costs}

The currency used is the 'Euro’ written as EUR or €.\\

Euro notes start from 5 € and go up to 200 €, whereas the coins have denominations of 1 cent, 2 cents, 5 cents, 10 cents, 20 cents, 50 cents, 1 €, and 2 €.\\

Average Cost of some items
\begin{itemize}
\item Business Lunch  25 €
\itemCup of Coffee  4 €
\itemFilled Sandwich  5 €
\itemMineral Water  2 €
\end{itemize}

Most stores accept credit and debit cards

\section*{Water}
The tap water in Vienna has a very high quality and is safe to drink.

\section*{Vienna Sights and Tourism Info Desk}

The Vienna Tourism Board will provide a tourism info desk directly at the conference venue from July 28 to 30, 2025 for your convenience. The provided services include:

\begin{itemize}

\item Tourist information about Vienna
\item Event details and recommendations
\item Assistance with hotel bookings
\item Various brochures about Vienna, including city maps in multiple languages

\end{itemize}

Feel free to also explore all attractive sights and events in Vienna around the time of the conference here.

The official Vienna City Card provides you with discounts for public transportation and tourist attractions. Visit \href{https://www.viennacitycard.at/en/} for more information.

\section*{Popular Destinations}
\begin{itemize}

\item 🎨Albertina (Art museum: Monet, Picasso and others, \href{https://www.albertina.at/en/home}

\item 🎨Kunsthistorisches Museum (Museum of Art History, very large collection: Tizian, Raffael, Caravaggio, Dürer, Bruegel, Rubens, Vermeer, Rembrandt, …) \href{https://www.khm.at/}

\item 🎡Riesenrad (Giant ferris wheel in the Prater amusement park) \href{https://wienerriesenrad.com/en/home-2}

\item ⛪Stephansdom (Cathedral, \href{https://www.stephanskirche.at/visit.php}

\item 🏰Schloss Schönbrunn (Castle, \href{https://www.schoenbrunn.at/en/visitor-information/how-to-get-here}

\item 🏰Schloss Belvedere (Castle and art museum, holds Klimt’s famous work “The Kiss” \href{https://www.belvedere.at/en})
\end{itemize}

\section*{Popular Shopping Areas}
\begin{itemize}

\item Mariahilfer Straße: \href{https://www.wien.info/de/sehen-erleben/shopping/shopping-rund-um-die-altstadt}

\item Kärntner Straße: \href{https://www.wien.info/en/see-do/shopping/old-city}

\end{itemize}
